%%%%%%%%%%%%%%%%%%%%%%%%%%%%%%%%%%%%%%%%%%%%%%%%%%%%%%%%%%
%
\section{SiCF et SiCP}
%
%%%%%%%%%%%%%%%%%%%%%%%%%%%%%%%%%%%%%%%%%%%%%%%%%%%%%%%%%%
%
Lorsque le programme est démarré en ligne de commande, il est possible de passer un certain nombre d'option. Elle sont communiqué au programme à l'aide du nom de l'option suivie d'un nombre. Par exemple pour démarrer SiCF avec un fond sombre et une discrétisation du temps égale à 0,00033 seconde :
\begin{center}
\texttt{./SiCF fond 17 dt 0.00033}
\end{center}
Pour démarrer SiCP avec un fond sombre, un nombre de pendules égale à 50 et trois solitons :
\begin{center}
\texttt{./SiCP fond 17 soliton 3 nombre 50}
\end{center}
%
%
\subsection{Résumé des options}
\begin{center}
\begin{tabular}{cccc}
option & valeur & clavier & commande \\
%\hline
{\texttt fond} & (fond>0 \& fond<255) &  & Couleur du fond \\
{\texttt soliton} & (soliton > -99 \& soliton < 99) & {\sf Y},{\sf H} & déphasage entre les extrémitées *\\
{\texttt dt} & (dt > 0.0 \& dt < DT\_MAX) &  & discrétisation du temps \\
{\texttt frequence} & () & {\sf P}, {\sf M} & fréquence du générateur \\
{\texttt dissipation} & () & {\sf E}, {\sf D} & dissipation \\
{\texttt equation} & (equation > 0 \& equation < 5) & {\sf F1}, {\sf F2}, {\sf F3}, {\sf F4} & choix de l'équation \\
{\texttt pause} & (pause > 5 || pause < 555) &  & temps de pause en ms \\
{\texttt duree} & () & {\sf F11}, {\sf F12} & Nombre d'évolution du système entre les affichages \\
{\texttt mode} & () & {\sf Entrée} & Mode -1 : Wait, 1 : Poll \\
{\texttt nombre} & (nombre > 0 \& nombre < 1099) &  & Nombre de pendules **\\
{\texttt aide} & () &  & Affiche l'aide \\
{\texttt help} & () &  & Affiche l'aide \\
\end{tabular}
\end{center}
* Initialise le déphasage entre le dernier pendule et le premier pendule dans le cas des conditions aux limites périodique.
** Spécifique à SiCP.
%
%
\subsection{Résumé du clavier SiCF et SiCP}
%
Le clavier permet de modifier les paramètres physiques. La fenêtre graphique doit être active, le terminal affiche les informations.
\begin{center}
\begin{tabular}{cccccccccc}
%\sffamily
%\rmfamily
\sf A &\sf Z &\sf E &\sf R &\sf T &\sf Y &\sf U &\sf I &\sf O &\sf P \\
Couplage & Masse & Dissip. & supprimD & Gravit. & Phi & Ampl. & impuls. & sinus & Fréquence \\
\sf Q &\sf S &\sf D &\sf F &\sf G &\sf H &\sf J &\sf K &\sf L &\sf M \\
moinsC & moinsM & plusD & formeD & moinsG & moins$\Phi$ & moinsA &  & carré & moinsF \\
\sf W &\sf X &\sf C &\sf V &\sf B &\sf N &  &  &  & \\
périodique & libres & fixe & ExtAbsD & libFix & fixLib &  &  &  & \\
\end{tabular}
\end{center}
\vspace{.3cm}
%
Les touches de fonctions donnent un certain nombre de contrôles et d'information:
%
\begin{center}
\begin{tabular}{ccccc ccccc cc}
\multicolumn{4}{|c|}{Contrôles} & \multicolumn{4}{c}{Information} & \multicolumn{4}{|c|}{Contrôles}\\
\sf F1 &\sf F2 &\sf F3 &\sf F4 &\sf F5 &\sf F6 &\sf F7 &\sf F8 &\sf F9 &\sf F10 &\sf F11 &\sf F12 \\
\multicolumn{4}{|c|}{Équation simulé (SiCF)} & \multicolumn{4}{c}{Énergie, graphe} & \multicolumn{4}{|c|}{Vitesse de la simulation}\\
\end{tabular}
\end{center}
%
Le choix de l'équation simulée est spécifique à SiCF. {\sf F5} dresse un bilan énergétique. {\sf F6} affiche les paramètres physiques du système
\begin{center}
\begin{tabular}{cccccc}
\sf F1 &\sf F2 &\sf F3 &\sf F4 &\sf F5 &\sf F6\\
Pendules & Harmoniques & Corde & asymétrique & Énergie & Système \\
\end{tabular}
\end{center}
%
F8 permet de modifier  le graphisme de SiCP. {\sf F9} et {\sf F12} modifient rapidement la vitesse de la simulation, {\sf F10} et {\sf F11} la modifie modéremment. La touche {\sf Entrée} change le mode avec ou sans attente, en mode avec attente, l'appuie sur une touche permet l'évolution du système.
\begin{center}
\begin{tabular}{cccccc}
\sf F8 &\sf F9 &\sf F10 &\sf F11 &\sf F12 & \sf Entrée \\
Support (SiCP) & -Sim & -Sim & +Sim & +Sim & mode\\
\end{tabular}
\end{center}
%
%
\subsection{Graphisme SiCP}
Cliquer et déplacer le pointeur de la souris permet de déplacer le point de vue de l'observateur. La touche {\sf F8} permet de supprimer/ajouter le support dans SiCP.
%
\subsection{Sauvegarde et initialisation dans SiCF}
%
Cette fonctionnalité nécessite la présence du répertoire donnee/enregistrement dans le répertoire de l'exécutable.
La touche majuscule permet d'accéder aux fonctions d'enregistrement et de réinitialisation des positions des pendules. Lorsque la touche majuscule est enfoncé, les touches {\sf a}, {\sf z}, {\sf e}, {\sf r}, {\sf t}, {\sf y}, {\sf u}, {\sf i}, {\sf o} et {\sf p} ré-initialisent la position de la corde dans différentes configurations préréglées. Les touches {\sf w}, {\sf x}, {\sf c}, {\sf v}, {\sf b} et {\sf n} enregistrent la position de la corde dans l'état actuel, les touches {\sf q}, {\sf s}, {\sf d}, {\sf f}, {\sf g} et {\sf h} réinitialisent la position de la corde dans les états enregistés.
%
\subsection{Détails des contrôles}
%
%\begin{itemize}[leftmargin=1cm, label=\ding{32}, itemsep=0pt]
%
\subsubsection{Équation simulée}
%
Il s'agit d'une spécificité de SiCF. Lorsque le mode asymétrique est activé, les touches {\sf Z} et {\sf S} ne change la masse de la corde que pour la moitié droite. 
%
\begin{itemize}[leftmargin=2cm, label=\ding{32}, itemsep=0pt]
\item 1: {\bf gravitation} forceRappel = sinus de la position du pendule
\item 2: {\bf linearisation} forceRappel = proportionnelle à la position du pendule
\item 3: {\bf corde vibrante} forceRappel = 0
\item 4: {\bf corde vibrante asymétrique} permet de changer la masse sur une demi-corde.
\end{itemize}
%
\subsubsection{Paramètres des pendules}
%
\begin{itemize}[label=\ding{32}, leftmargin=2cm, itemsep=0pt]
\item {\bf Couplage} : {\sf A}, {\sf Q} : Augmente, diminue le couplage entre les pendules.
\item {\bf Masse} : {\sf Z}, {\sf S} :  Augmente, diminue la masse des pendules.
\item {\bf Dissipation} : {\sf E}, {\sf D} :  Augmente, diminue les frottements visqueux.
\item {\bf Gravitation} : {\sf T}, {\sf G} :  Augmente, diminue l'accélération de la gravitation.
%\item {\bf } : \sf{} : 
\end{itemize}
%
%
\subsubsection{Forme de la dissipation}
%
La touche {\sf V} supprime les frottements sauf pour les derniers pour lesquels les frottements s'accroissent. Ceci permet d'obtenir une extrémité "absorbante", permettant la simulation d'une corde infinie.
%
\begin{itemize}[label=\ding{32}, leftmargin=2cm, itemsep=0pt]
\item {\bf Supprimer} : {\sf E} : Supprime les frottements visqueux.
\item {\bf Former} : {\sf F} : Active les frottements visqueux sur toute la chaîne.
\item {\bf Absorber} : {\sf V} : Active les frottements visqueux sur la fin de la chaîne, crée une extrémité absorbante.
\end{itemize}
%
%
\subsubsection{Conditions aux limites}
%
\begin{itemize}[label=\ding{32}, leftmargin=2cm, itemsep=0pt]
\item {\bf Périodique} : {\sf W} : Le dernier pendule est couplé au premier.
\item {\bf Libres} : {\sf X} : Les deux extrémités sont libres.
\item {\bf Fixes} : {\sf C} : Les deux extrémités sont fixes.
\item {\bf libre-fixe} : {\sf B} : Le premier pendule est libre et le dernier pendule est fixe.
\item {\bf fixe-libre} : {\sf N} : Le premier pendule est fixe et le dernier pendule est libre.
\end{itemize}
%
%
\subsubsection{Moteur premier pendule}
%
\begin{itemize}[label=\ding{32}, leftmargin=2cm, itemsep=0pt]
\item {\bf Impulsion} : {\sf I} : Crée une impulsion.
\item {\bf Sinus} : {\sf O} : Active, désactive le moteur sinusoïdale.
\item {\bf Sinus} : {\sf L} : Active le moteur carré.
\item {\bf Amplitude} : {\sf U, J} :  Augmente, diminue l'amplitude du moteur.
\item {\bf Fréquence} : {\sf P, M} :  Augmente, diminue la fréquence du moteur.
\end{itemize}
%
%
\subsubsection{Moteur Josephson}
%
\begin{itemize}[label=\ding{32}, leftmargin=2cm, itemsep=0pt]
\item {\bf Activation} : $\rightarrow$ : Crée, supprime un courant josephson.
\item {\bf Amplitude} : $\mathtt{\uparrow}$, $\mathsf{\downarrow}$ : Augmente, diminue le courant.
\item {\bf Sens} : {\sf $\leftarrow$} : Inverse le sens du courant josephson.
%\item {\bf } : \sf{} : 
\end{itemize}
%
%
\subsubsection{Contrôle de la simulation}
%
{\sf F9} et {\sf F12} modifient rapidement la vitesse de la simulation, {\sf F10} et {\sf F11} la modifient modéremment. La touche {\sf Entrée} change le mode avec ou sans attente, en mode avec attente, l'appuie sur une touche permet l'évolution du système.
%
\begin{itemize}[label=\ding{32}, leftmargin=2cm, itemsep=0pt]
\item {\bf Mode} : {\sf Entrée} : Change le mode de la simulation : évolution automatique ou pas à pas.
\item {\bf Accélèrer} : {\sf 11} et {\sf F12} : Accélère la simulation.
\item {\bf Ralentir} : {\sf F9} et {\sf F10} : Ralentit la simulation.
\end{itemize}
%
%
\subsubsection{Information}
\begin{itemize}[label=\ding{32}, leftmargin=2cm, itemsep=0pt]
\item {\bf Énergie} : {\sf F5} : Information énergétique de la chaîne.
\item {\bf Système} : {\sf F6} : Affiche les paramètres physiques du système.
%\item {\bf } : \sf{F7} : 
%\item {\bf } : \sf{F8} : 
\end{itemize}
%\end{itemize}
%
%\item {\bf } : \sf{} : 
%%%%%%%%%%%%%%%%%%%%%%%%%%%%%%%%%%%%%%%%%%%%%%%%%%%%%%%%%
