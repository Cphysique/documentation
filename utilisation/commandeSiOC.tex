%%%%%%%%%%%%%%%%%%%%%%%%%%%%%%%%%%%%%%%%%%%%%%%%%%%%%%%%%%
%
\section{SiCF et SiCP}
%
%%%%%%%%%%%%%%%%%%%%%%%%%%%%%%%%%%%%%%%%%%%%%%%%%%%%%%%%%%
%
Lorsque le programme est démarré en ligne de commande, il est possible de passer un certain nombre d'options. Ces options sont communiquées au programme à l'aide du nom de l'option suivie d'un nombre. Par exemple pour démarrer SiCF avec une discrétisation spatiale égale à 16 et une discrétisation du temps égale à 0,00033 seconde :
\begin{center}
\texttt{./SiCF2 nombre 16 dt 0.00033}
\end{center}
Pour démarrer SiCP avec un nombre de pendules égale à 50 et trois solitons :
\begin{center}
\texttt{./SiCP2 soliton 3 nombre 50}
\end{center}
%
Pour démarrer SiCP en exploitant plusieurs options :
\begin{center}
\texttt{./SiCP2 support 0 pause 7 duree 998 dt 0.0027 nombre 777 soliton 18}
\end{center}
%
\subsection{Résumé des options}
\begin{center}
\begin{tabular}{cccc}
option & valeur & clavier & commande \\
%\hline
{\texttt soliton} & (soliton > -99 \& soliton < 99) & {\sf y},{\sf h} & déphasage entre les extrémitées *\\
{\texttt dt} & (dt > 0.0 \& dt < DT\_MAX) &  & discrétisation du temps \\
{\texttt duree} & () & {\sf F9}, {\sf F10}, {\sf F11}, {\sf F12} & rythme de la simulation \\
{\texttt mode} & () & {\sf Entrée} & Mode -1 : Wait, 1 : Poll \\
{\texttt nombre} & (nombre > 0 \& nombre < 1099) &  & Nombre de pendules\\
{\texttt nombre} & (nombre >= 8 \& nombre =< 1024) &  & Discrétisation spatiale\\
{\texttt aide} & () &  & Affiche l'aide \\
{\texttt help} & () &  & Affiche l'aide \\
\end{tabular}
\end{center}

{\it
* Initialise le déphasage entre le dernier pendule et le premier pendule dans le cas des conditions aux limites périodique.

}
%
