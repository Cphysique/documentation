\section{Installation des simulateurs}
Cette section traite de l'installation des simulateurs SiGP, SiCF et SiCP sur un système d'exploitation de type debian. Le téléchargement se fait avec un navigateur internet, la compilation et l'exécution se font dans un terminal. L'installation des outils de compilation nécessite les privilèges du super-utilisateur.
\begin{itemize}[leftmargin=1cm, label=\ding{32}, itemsep=0pt]
\item {\bf Installation des outils de compilation}
Avec les droits du super-utilisateur
	\begin{itemize}[leftmargin=1cm, label=\ding{32}, itemsep=0pt]
	\item \texttt{apt-get install gcc make libsdl-dev}
	\item Pour les versions 2 des simulateurs, installer la librairie SDL2 :
	\item \texttt{apt-get install libsdl2-dev}
	\end{itemize}
\item {\bf Téléchargement des sources}
	\begin{itemize}[leftmargin=1cm, label=\ding{32}, itemsep=0pt]
	\item Télécharger les fichiers \texttt{.zip} sur github
		\begin{itemize}[leftmargin=1cm, label=\ding{32}, itemsep=0pt]
		\item \texttt{https://github.com/runigo/SiCP/archive/master.zip}
		\item \texttt{https://github.com/runigo/SiCF/archive/master.zip}
		\item \texttt{https://github.com/runigo/SiGP/archive/master.zip}
		\end{itemize}
	\item Décompresser les fichiers \texttt{.zip}
		\begin{itemize}[leftmargin=1cm, label=\ding{32}, itemsep=0pt]
		\item \texttt{unzip SiCP-master.zip}
		\item \texttt{unzip SiCF-master.zip}
		\item \texttt{unzip SiGP-master.zip}
		\end{itemize}
	\end{itemize}
\item {\bf Compilation}
	\begin{itemize}[leftmargin=1cm, label=\ding{32}, itemsep=0pt]
	\item La commande \texttt{make} dans le répertoire des sources produit un fichier exécutable :
		\begin{itemize}[leftmargin=1cm, label=\ding{32}, itemsep=0pt]
		\item \texttt{SiCP} pour SiCP
		\item \texttt{SiCF} pour SiCF
		\item \texttt{SiGP} pour SiGP
		\end{itemize}
	\end{itemize}
%
\item {\bf Exécution}
	\begin{itemize}[leftmargin=1cm, label=\ding{32}, itemsep=0pt]
	\item En ligne de commande, avec d'éventuelles options
		\begin{itemize}[leftmargin=1cm, label=\ding{32}, itemsep=0pt]
		\item \texttt{./SiCP [OPTION]}
		\item \texttt{./SiCF [OPTION]}
		\item \texttt{./SiGP [OPTION]}
		\end{itemize}
	\item La fenêtre graphique donne une représentation de la simulation,
	\item Le terminal affiche les informations.
	\end{itemize}
\end{itemize}

%%%%%%%%%%%%%%%%%%%%%%%%%%%%%%%%%%%%%%%%%%%%%%%%%%%%%%%%%%%%%%%%%%%%%%%%%%%%%%%%%%%%%%%%%%%%%
