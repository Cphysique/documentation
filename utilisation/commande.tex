
%%%%%%%%%%%%%%%%%%%%%%%%%%%%%%%%%%%
\section{Commandes des simulateurs}
%%%%%%%%%%%%%%%%%%%%%%%%%%%%%%%%%%%

Cette section traite des interactions entre le programme et l'utilisateur.

\subsection{Commande du simulateur de gaz parfait SiGP}

Lorsque le programme est démarré en ligne de commande, il est possible de passer un certain nombre d'option. Elle sont communiqué au programme à l'aide du nom de l'option suivies d'un nombre. Par exemple pour démarrer SiGP avec un fond sombre, deux fluxons et une discrétisation du temps égale à 0,033 seconde :

\begin{center}
\texttt{./SiGP cf 17 df 2 dt 0.033}
\end{center}

\subsubsection{Options et commande du clavier}

\begin{center}
\begin{tabular}{cccc}
option & valeur & clavier & commande \\
\hline
pause & 5 < d < 555 &  & pause entre les affichages en ms \\
%duree & 1 < d < 99 & flèches & nombre d'évolution du système entre les affichages \\
duree & 1 < d < 99 & flèches & vitesse de la simulation \\
cloison & -3 < d < 3 & \texttt{w-n} & cloison si <> 0 \\
thermostat & -2 < d < 2 & \texttt{o} , \texttt{l} & système isolé si = 0 \\
vitesse & 0.03 < f < 33.3 &  & Vitesse initiale \\
temperature & 0.0000003 < f < 90 000 & \texttt{p}, \texttt{m} & Température thermostat \\
gauche & 0.0000003 < f < 90 000 & \texttt{u}, \texttt{j} & Thermostat gauche \\
droite & 0.0000003 < f < 90 000 & \texttt{i}, \texttt{k} & Thermostat droite \\
\end{tabular}
\end{center}

\subsubsection{Cloison centrale}

	{\it Option} : 

\begin{itemize}[leftmargin=1cm, label=\ding{32}, itemsep=0pt]
\item 0: {\bf Pas de cloison}
\item 1: {\bf Cloison percée.}
\item 2: {\bf Cloison}
\item -1: {\bf Cloison percée et démon de maxwell.}
\item -2: {\bf Cloison et démon de maxwell}
\end{itemize}

	{\it Clavier} : 

\begin{itemize}[leftmargin=1cm, label=\ding{32}, itemsep=0pt]
\item w: {\bf Pas de cloison}
\item x: {\bf Cloison percée}
\item c: {\bf Cloison}
\item b: {\bf Cloison percée et démon de maxwell.}
\item n: {\bf Cloison et démon de maxwell.}
\end{itemize}

\subsubsection{Thermostat}

	{\it Option} : 

\begin{itemize}[leftmargin=1cm, label=\ding{32}, itemsep=0pt]
\item 0: {\bf Système isolé}
\item 1: {\bf Température gauche-droite identiques}
\item 2: {\bf Température gauche-droite différentes}
\end{itemize}

	{\it Clavier} : 

\begin{itemize}[leftmargin=1cm, label=\ding{32}, itemsep=0pt]
\item o: {\bf Système isolé}
\item l: {\bf Température gauche-droite identiques}
\item l: {\bf Température gauche-droite différentes}
\end{itemize}

\texttt{mo} : {\bf Mode -1 : Wait, 1 : Poll} %optionsMode(option, opt[i+1]);
  (mode == 1 || mode == -1)

\normalsize


\subsection{Option de la ligne de commande de SiCP 1.3}
Lorsque le programme est démarré en ligne de commande, il est possible de passer un certain nombre d'option. Elle sont communiqué au programme à l'aide de deux lettres suivies d'un nombre. Par exemple pour démarrer SiCP avec un fond sombre, deux fluxons et une discrétisation du temps égale à 0,0033 seconde :
\begin{center}
\texttt{./SiCP fond 17 soliton 2 dt 0.0033}
\end{center}
Les options possibles sont :
\begin{itemize}[leftmargin=2cm, label=\ding{32}, itemsep=3pt]
\item {\large \texttt{nombre}} : {\bf Nombre de pendules.} %optionsDephasage(option, opt[i+1]);
  (nombre > 2 \&\& nombre < NOMBRE\_MAX). Initialise le nombre de pendules couplés.
\item {\large \texttt{soliton}} : {\bf déphasage entre les extrémitées.} %optionsDephasage(option, opt[i+1]);
  (soliton > -99 \&\& soliton < 99). Initialise le déphasage entre le dernier pendule et le premier pendule dans le cas des conditions aux limites périodique.
\item {\large \texttt{equation}} : {\bf choix de l'équation} %optionsEquation(option, opt[i+1]);
  (equation > 0 \&\& equation < 5).  Calcul de la FORCE DE RAPPEL (gamma est négatif)
\begin{itemize}[leftmargin=1cm, label=\ding{32}, itemsep=0pt]
\item 1: {\bf gravitation} forceRappel = sinus de la position du pendule
\item 2: {\bf linearisation} forceRappel = proportionnelle à la position du pendule
\item 3: {\bf corde vibrante} forceRappel = 0
\item 4: {\bf corde vibrante asymétrique} forceRappel = 0
\end{itemize}
%\item default:// corde vibrante forceRappel = 0.0;
\item {\large \texttt{fond}} : {\bf Couleur du fond} %optionsFond(option, opt[i+1]);
  (fond>0 \&\& fond<255)
%\item {\Large \texttt{th}} Deux threads %optionsThread(option, opt[i+1]);
  %(thread==1 || thread==0)
%\item {\large \texttt{mo}} : {\bf Mode -1 : Wait, 1 : Poll} %optionsMode(option, opt[i+1]);
%  (mode == 1 || mode == -1)
\item {\large \texttt{pause}} : {\bf temps de pause en ms} %optionsPause(option, opt[i+1]);
  (pause > 5 || pause < 555)
\item {\large \texttt{dt}} : {\bf discrétisation du temps} %optionsDt(option, opt[i+1]);
  (dt > 0.0 \&\& dt < DT\_MAX)
\end{itemize}

\subsection{Option de la ligne de commande de SiCP64 1.1}
Lorsque le programme est démarré en ligne de commande, il est possible de passer un certain nombre d'option. Elle sont communiqué au programme à l'aide du nom de l'option suivies d'un nombre. Par exemple pour démarrer SiCP64 avec un fond sombre, deux fluxons et une discrétisation du temps égale à 0,033 seconde :
\begin{center}
\texttt{./SiCP64 cf 17 df 2 dt 0.033}
\end{center}
Les options possibles sont :
\begin{itemize}[leftmargin=2cm, label=\ding{32}, itemsep=3pt]
\item {\large \texttt{df}} : {\bf déphasage entre les extrémitées.} %optionsDephasage(option, opt[i+1]);
  (fluxon > -99 \&\& fluxon < 99). Initialise le déphasage entre le dernier pendule et le premier pendule dans le cas des conditions aux limites périodique.
\item {\large \texttt{eq}} : {\bf choix de l'équation} %optionsEquation(option, opt[i+1]);
  (equation > 0 \&\& equation < 5).  Calcul de la FORCE DE RAPPEL (gamma est négatif)
\begin{itemize}[leftmargin=1cm, label=\ding{32}, itemsep=0pt]
\item 1: {\bf gravitation} forceRappel = sinus de la position du pendule
\item 2: {\bf linearisation} forceRappel = proportionnelle à la position du pendule
\item 3: {\bf corde vibrante} forceRappel = 0
\item 4: {\bf corde vibrante asymétrique} forceRappel = 0
\end{itemize}
%\item default:// corde vibrante forceRappel = 0.0;
\item {\large \texttt{cf}} : {\bf Couleur du fond} %optionsFond(option, opt[i+1]);
  (fond>0 \&\& fond<255)
%\item {\Large \texttt{th}} Deux threads %optionsThread(option, opt[i+1]);
  %(thread==1 || thread==0)
\item {\large \texttt{mo}} : {\bf Mode -1 : Wait, 1 : Poll} %optionsMode(option, opt[i+1]);
  (mode == 1 || mode == -1)
\item {\large \texttt{po}} : {\bf temps de pause en ms} %optionsPause(option, opt[i+1]);
  (pause > 5 || pause < 555)
\item {\large \texttt{dt}} : {\bf discrétisation du temps} %optionsDt(option, opt[i+1]);
  (dt > 0.0 \&\& dt < DT\_MAX)
\end{itemize}

\subsection{Commande du clavier}

\subsubsection{Résumé SiCP64 1.1}
\begin{center}
\begin{tabular}{cccccccccc}
%\sffamily
%\rmfamily
\sf A &\sf Z &\sf E &\sf R &\sf T &\sf Y &\sf U &\sf I &\sf O &\sf P \\
Couplage & Masse & Dissip. & supprimD & Gravit. & Phi & Sym. & impuls. & sinCar & fréquence \\
\sf Q &\sf S &\sf D &\sf F &\sf G &\sf H &\sf J &\sf K &\sf L &\sf M \\
moinsC & moinsM & plusD & formeD & moinsG & moins$\Phi$ & moinsS & Ampl. & moinsA & moinsF \\
\sf W &\sf X &\sf C &\sf V &\sf B &\sf N &  &  &  & \\
périodique & libres & fixe & ExtAbsD & libFix & fixLib &  &  &  & \\
\end{tabular}
\end{center}
\vspace{.3cm}
\begin{center}
\begin{tabular}{ccccc ccccc cc}
%\sffamily
%\rmfamily
\multicolumn{4}{|c|}{Contrôles} & \multicolumn{4}{c}{Information} & \multicolumn{4}{|c|}{SiCP64}\\
\sf F1 &\sf F2 &\sf F3 &\sf F4 &\sf F5 &\sf F6 &\sf F7 &\sf F8 &\sf F9 &\sf F10 &\sf F11 &\sf F12 \\
Mode & -Sim & +Sim &  & Énergie  &  &  &   & Altitude & gauche & droite & Altitude \\
\sf Entrée &\sf - &\sf + &\sf  &\sf  &\sf  &\sf  &\sf  &\sf  &\sf  \\
Mode & -Sim & +Sim & & & & & & & \\
\end{tabular}
\end{center}
\subsubsection{Résumé SiCP 1.3}
\begin{center}
\begin{tabular}{cccccccccc}
%\sffamily
%\rmfamily
\sf A &\sf Z &\sf E &\sf R &\sf T &\sf Y &\sf U &\sf I &\sf O &\sf P \\
Couplage & Masse & Dissip. & supprimD & Gravit. & Phi & Sym. & impuls. & sinCar & fréquence \\
\sf Q &\sf S &\sf D &\sf F &\sf G &\sf H &\sf J &\sf K &\sf L &\sf M \\
moinsC & moinsM & plusD & formeD & moinsG & moins$\Phi$ & moinsS & Ampl. & moinsA & moinsF \\
\sf W &\sf X &\sf C &\sf V &\sf B &\sf N &  &  &  & \\
périodique & libres & fixe & ExtAbsD & libFix & fixLib &  &  &  & \\
\end{tabular}
\end{center}
\vspace{.3cm}
\begin{center}
\begin{tabular}{ccccc ccccc cc}
%\sffamily
%\rmfamily
\multicolumn{4}{|c|}{Contrôles} & \multicolumn{4}{c}{Information} & \multicolumn{4}{|c|}{SiCP}\\
\sf F1 &\sf F2 &\sf F3 &\sf F4 &\sf F5 &\sf F6 &\sf F7 &\sf F8 &\sf F9 &\sf F10 &\sf F11 &\sf F12 \\
Pendules & Harmoniques & Corde & asymétrique & Énergie &  &  &   &  &  &  &  \\
\sf Entrée &\sf - &\sf + &\sf  &\sf  &\sf  &\sf  &\sf  &\sf  &\sf  \\
Mode & -Sim & +Sim & & & & & & & \\
\end{tabular}
\end{center}
%
\subsubsection{Paramètres de la chaîne}
%
\begin{itemize}[label=\ding{32}, leftmargin=2cm]
\item {\bf Couplage} : \texttt{a, q} : Augmente, diminue le couplage entre les pendules.
\item {\bf Masse} : \texttt{z, s} :  Augmente, diminue la masse des pendules.
\item {\bf Dissipation} : \texttt{e, d} :  Augmente, diminue les frottements visqueux.
\item {\bf Gravitation} : \texttt{t, g} :  Augmente, diminue l'accélération de la gravitation.
%\item {\bf } : \texttt{} : 
\end{itemize}
%
\subsubsection{Forme de la dissipation}
%
\begin{itemize}[label=\ding{32}, leftmargin=2cm]
\item {\bf Supprimer} : \texttt{e} : Supprime les frottements visqueux.
\item {\bf Former} : \texttt{f} : Active les frottements visqueux sur toute la chaîne.
\item {\bf Absorber} : \texttt{v} : Active les frottements visqueux sur la fin de la chaîne, crée une extrémité absorbante.
\end{itemize}
%
\subsubsection{Conditions aux limites}
%
\begin{itemize}[label=\ding{32}, leftmargin=2cm]
\item {\bf Périodique} : \texttt{w} : Le dernier pendule est couplé au premier.
\item {\bf Libres} : \texttt{x} : Les deux extrémités sont libres.
\item {\bf Fixes} : \texttt{c} : Les deux extrémités sont fixes.
\item {\bf libre-fixe} : \texttt{b} : Le premier pendule est libre et le dernier pendule est fixe.
\item {\bf fixe-libre} : \texttt{n} : Le premier pendule est fixe et le dernier pendule est libre.
\end{itemize}
%
\subsubsection{Moteur premier pendule}
%
\begin{itemize}[label=\ding{32}, leftmargin=2cm]
\item {\bf Impulsion} : \texttt{i} : Crée une impulsion.
\item {\bf Sinus} : \texttt{o} : Active, désactive le moteur sinusoïdale.
\item {\bf Amplitude} : \texttt{k, l} :  Augmente, diminue l'amplitude du moteur.
\item {\bf Fréquence} : \texttt{p, m} :  Augmente, diminue la fréquence du moteur.
\end{itemize}
%
\subsubsection{Moteur Josephson}
%
\begin{itemize}[label=\ding{32}, leftmargin=2cm]
\item {\bf Activation} : \texttt{$\rightarrow$} : Crée, supprime un courant josephson.
\item {\bf Amplitude} : \texttt{$\uparrow, \downarrow$} : Augmente, diminue le courant.
\item {\bf Sens} : \texttt{$\leftarrow$} : Inverse le sens du courant josephson.
%\item {\bf } : \texttt{} : 
\end{itemize}
%
\subsubsection{Contrôle de la simulation}
%
\begin{itemize}[label=\ding{32}, leftmargin=2cm]
\item {\bf Mode} : \texttt{Entrée} ou \texttt{F1} : Change le mode de la simulation : temps réèl ou pas à pas.
\item {\bf Accélèrer} : \texttt{+} ou \texttt{F2} : Accélère la simulation.
\item {\bf Ralentir} : \texttt{-} ou \texttt{F3} : Ralentit la simulation.
%\item {\bf } : \texttt{} : 
\end{itemize}
%
\subsubsection{Information}
\begin{itemize}[label=\ding{32}, leftmargin=2cm]
\item {\bf Énergie} : \texttt{F5} : Information énergetique de la chaîne.
%\item {\bf } : \texttt{F6} : 
%\item {\bf } : \texttt{F7} : 
%\item {\bf } : \texttt{F8} : 
\end{itemize}
%
\subsubsection{Graphisme SiCP64}
%
\begin{itemize}[label=\ding{32}, leftmargin=2cm]
\item {\bf Altitude} : \texttt{F9} : Diminue l'altitude du point de vue.
\item {\bf Rotation} : \texttt{F10} : Tourne la chaîne vers la gauche.
\item {\bf Rotation} : \texttt{F11} : Tourne la chaîne vers la droite.
\item {\bf Altitude} : \texttt{F12} : Augmente l'altitude du point de vue.
\end{itemize}
%
\subsection{Commande de la souris}
%
\subsubsection{Graphisme SiCP64}
%
Cliquer et déplacer le pointeur de la souris permet de déplacer le point de vue de l'observateur.
\subsection{Enregistrement de la position}

\subsubsection{Dans SiCP64}
%
La touche majuscule permet d'accéder aux fonctions d'enregistrement et de réinitialisation des positions des pendules.
%\item {\bf } : \texttt{} : 
%%%%%%%%%%%%%%%%%%%%%%%%%%%%%%%%%%%%%%%%%%%%%%%%%%%%%%%%%ù
