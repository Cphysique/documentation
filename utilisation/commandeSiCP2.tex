%%%%%%%%%%%%%%%%%%%%%%%%%%%%%%%%%%%%%%%%%%%%%%%%%%%%%%%%%%
%
\section{SiCP2}
%
%%%%%%%%%%%%%%%%%%%%%%%%%%%%%%%%%%%%%%%%%%%%%%%%%%%%%%%%%%
%
SiCP2 possède une interface graphique permettant de modifier les paramètres à l'aide de la souris. Lorsque le programme est démarré en ligne de commande, il est toujours possible de passer un certain nombre d'options. La syntaxe est identique à SiCP :
\begin{center}
\texttt{./SiCP2 support 0 dt 0.0027 nombre 177 soliton 7}
\end{center}
L'utilisation du clavier est identique à SiCP.
%
%
%
\subsection{Panneau de droite}
%
Ce panneau possède cinq boutons rotatifs et 17 boutons "radio". La rotation s'effectue en plaçant le pointeur de la souris au centre dun de ces boutons et en actionnant la molette. La sélection d'un bouton radio s'effectue en cliquant sur celui-ci. 
%
%
%
\subsection{Panneau du bas}
%
%
%

%
%
%
\subsection{Panneau central}
%
%
%
%
\subsection{Clavier}
%
% V1
%Le clavier permet de modifier les paramètres physiques. La fenêtre graphique doit être active, le terminal affiche les informations.
\footnotesize
\normalsize
%
\subsection{Graphisme SiCP}
Cliquer et déplacer le pointeur de la souris permet de déplacer le point de vue de l'observateur. La touche {\sf F8} permet de supprimer/ajouter le support dans SiCP.
%
\subsection{Sauvegarde et ré-initialisation dans SiCF}
%
Cette fonctionnalité nécessite la présence du répertoire {\texttt donnee/enregistrement} dans le répertoire de l'exécutable.
La touche majuscule permet d'accéder aux fonctions d'enregistrement et de ré-initialisation des positions des pendules.

Lorsque la touche majuscule est enfoncé, les touches {\sf A}, {\sf Z}, {\sf E}, {\sf R}, {\sf T}, {\sf Y}, {\sf U}, {\sf I}, {\sf O} et {\sf P}, ainsi que les touches {\sf J}, {\sf K}, {\sf L} et {\sf M} ré-initialisent la position de la corde dans différentes configurations préréglées.

Les touches {\sf W}, {\sf X}, {\sf C}, {\sf V}, {\sf B} et {\sf N} enregistrent la position de la corde dans l'état actuel, les touches {\sf Q}, {\sf S}, {\sf D}, {\sf F}, {\sf G} et {\sf H} réinitialisent la position de la corde dans ces états enregistés.
%
\subsubsection{Fonction élémentaire}
%
\begin{center}
\begin{tabular}{cc cc}%\multicolumn{4}{|c|}{}\\
Touche & fonction \\
A & nulle &\\
Z & impulsion &\\
E & triangle&\\
R & triangle&\\
T & carré &\\
Y & carré &\\
\end{tabular}
\end{center}
%
\subsubsection{Quanton}
%
\begin{center}
\begin{tabular}{cc cc}%\multicolumn{4}{|c|}{}\\
Touche & fonction \\
U, J & impulsion &\\
I, K & impulsion &\\
O, L & quanton &\\
P, M & quanton &\\
\end{tabular}
\end{center}

\subsubsection{Fichiers de ré-initialisation}
Les fichiers de ré-initialisation se trouvent dans le répertoire {\texttt donnee/enregistrement}. Ils peuvent être édités. Le nom de ces fichiers doit être respecté afin de pouvoir être ouvert par le programme (ces noms sont utilisés par {\texttt donnees/fichier.c}).
%
%\end{itemize}
%
%\item {\bf } : \sf{} : 
%%%%%%%%%%%%%%%%%%%%%%%%%%%%%%%%%%%%%%%%%%%%%%%%%%%%%%%%%
