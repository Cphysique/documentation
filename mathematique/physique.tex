%
%%%%%%%%%%%%%%%%%%%%%
\section{La physique}
%%%%%%%%%%%%%%%%%%%%%
%
\subsection{La relation fondamentale de la dynamique}
%
Somme des forces = m a
%
\subsection{L'équation de Sine-Gordon \cite{sine-gordon}}
%
C'est une équation différentielle du second ordre, non linéaire, à deux variables.
%
\[
\frac{\partial^2\theta}{\partial t^2} - c^2 \frac{\partial^2\theta}{\partial x^2} + \omega _0 ^2 \sin \theta = 0
\]
\subsection{La chaine de pendule}
%
La chaîne de pendule est constituée d'une série de pendules pesants. Chaque pendule étant couplé avec ses deux plus proches voisins à l'aide d'un fil de torsion.
%
\subsection{La corde vibrante}
%
\subsection{Les chocs de particules}
On applique les lois de conservations de l'énergie et de l'impulsion :
\[
\sum m_i{v'_i}^2=\sum m_iv_i^2
\hspace{2cm}
\sum m_i\overrightarrow{v'}_i=\sum m_i\overrightarrow{v}_i
\]
\begin{center}
$\sum m_i{v'_i}^2=\sum m_iv_i^2$
\hspace{2cm}
$\sum m_i\overrightarrow{v'}_i=\sum m_i\overrightarrow{v}_i$
\end{center}
\begin{center}
$m_i\mathbf{v'}_i^2=m_i\mathbf{v}_i^2$
\hspace{2cm}
$m_i\mathbf{v'}_i=m_i\mathbf{v}_i$
\end{center}
\subsubsection{Chocs de deux particules}
La conservations de l'énergie donne :
\[
m_1{v'_1}^2 + m_2{v'_2}^2=m_1v_1^2 + m_2v_2^2
\]
La conservations de l'impulsion donne :
\[
m_1\overrightarrow{v'}_1 m_2\overrightarrow{v'}_2=m_1\overrightarrow{v}_1 m_2\overrightarrow{v}_2
\]
Dans le référentiel du centre de masse : 
\[
(m_1 + m_2)\overrightarrow{v}=m_1\overrightarrow{v}_1 + m_2\overrightarrow{v}_2
\]
%
\subsection{Équation de Schrödinger}
%

%%%%%%%%%%%%%%%%%%%%%%%%%%%%%%%%%%%%%%%%%%%%
%%%%%%%%%%%%%%%%%%%%%%%%%%%%%%%%%%%%%%%%%%%%
\begin{itemize}[leftmargin=2cm]
\item \texttt{gras} : texte
\end{itemize}



%%%%%%%%%%%%%%%%%%%%%%%%%%%%%%%%%%%%%%%%%%%%%%%%%%%%%%%%%%%%%%%%%%%%%%%%%%%%%%%%%%%%%
