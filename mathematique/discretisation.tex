
%%%%%%%%%%%%%%%%%%%%%
\section{Discrétisation}
%%%%%%%%%%%%%%%%%%%%%

\subsection{Discrétisation des dérivées}
%
\subsubsection{Dérivé symétrisée}
La symétrisation de la dérivé permet de retrouver l'algorithme de Verlet. Une tel symétrisation permet d'observer la conservation de l'énergie lorsque l'on supprime les frottements visqueux.
\[
v=\frac{x(t+\dt)-x(t-\dt)}{2\dt}
\]
\[
a=\frac{x(t+2\dt)-x(t)-x(t)+x(t-2\dt)}{4\dt^2}
\]
%
\subsubsection{Changement de variable}
Simplifiant l'expression de l'accélération, le changement de variable dt' = 2 dt donne une expression disymétrique de la vitesse, cette dernière est utilisée pour l'évaluation des forces de viscosité.
\[
a=\frac{x(t+\dt)-2x(t)+x(t-\dt)}{\dt^2}
\]
\[
v=\frac{x(t)-x(t-\dt)}{\dt}
\]

\subsection{Discrétisation de la relation fondamentale de la dynamique}

Somme des forces = m a

Considérant que la variable x est un angle en radian, l'analyse dimentionnelle de la relation fondamentale de la dynamique donne alors

\[
-\dt^2mg\sin{x(t)}  -  \dt^2 k.l.D.x(t)  -  dt.a.l ( x(t) - x(t-dt) )  =  m.l ( x(t+dt) - 2x(t) + x(t-dt) )
\]
\[
\mathrm{
 T^2MLT^{-2}         T^2MT^{-2}L        TMT^{-1}L       =       ML
}
\]

soit

\[
- \dt^2\frac{g}{l}\sin{x(t)}-\dt^2\frac{kD}{m} x(t) - \dt\frac{a}{m}(x(t)-x(t-\dt))=(x(t+\dt)-2x(t)+x(t-\dt))
\]
\subsubsection{Variables réduites}

Les variables réduites sont sans dimension . Elles prennent en compte la discrétisation du temps. Le signe prend en compte le caractère « de rappel » des forces.

\[
\texttt{alpha} =  - \frac{a}{m}\dt
\]
\[
\texttt{gamma} =  - \frac{g}{l}\dt^2
\]
\[
\texttt{kapa} =  - \frac{k}{m}\dt^2
\]

Définition

définition simplifié des variables numériques

\[
force[i] = gamma.sinx + kapa.Dx + alpha.dx
\]

définition des variables numériques de SiCP 1.0

forceTotale = gamma sinx + kapaprécédent  Dxprécédent +  kapa Dxsuivant + alpha dx

Simulation Numérique


\[
	( x(t+\dt) - 2x(t) + x(t-\dt) ) = force[i]
\]

\[
	soit	x(t+\dt) = 2 x(t) - x(t-\dt) + force[i]
\]


Période égale à une seconde
		Période = 2pi sqrt(l/g) = 1			$g/l . x = - d2x / \dt^2 $
		g/l=4pi2 = 39,478				x = cos( t √ g/l )

				longueur = 0,25

Empiriquement, un affichage graphique par 30 ms, est obtenue avec delay(25)
 (Ce 25 n'a rien à voir avec la longueur précédente... )

		$\dt * duree = delay
		\dt * 100 = 0,03 		\dt = 0,0003$


\subsubsection{Limite infinie}

	pendule de  « précédent »  à  « - nombre * 5 / 6 » ,
		dissipation de 10  à 1 

	pendule précédents
		dissipation = 0,0



\subsubsection{Interaction dans SiG (Simulateur de gravitation)}

	$a=d2f/m\dt^2.=force[i]$

$x(t+\dt)=2.x(t)-x(t-\dt)+force$

nouveau[i]=2.actuel[i]-ancien[i]+force[i]

force[i]=champ[i].q[i]/m[i]

champ[j]=Sq[i].(r[i]-r[j])/|r[i]-r[j]|3.

ch[j]=Sq[i].(position-corps)...

Si les charges sont opposées,

	champ[j]=Sq[i].((r[i]-r[j])/|r[i]-r[j]|3.-(r[i]-r[j])/|r[i]-r[j]|5.

%\subsection{L'équation de Schrödinger}


%%%%%%%%%%%%%%%%%%%%%%%%%%%%%%%%%%%%%%%%%%%%
%%%%%%%%%%%%%%%%%%%%%%%%%%%%%%%%%%%%%%%%%%%%
%%%%%%%%%%%%%%%%%%%%%%%%%%%%%%%%%%%%%%%%%%%%%%%%%%%%%%%%%%%%%%%%%%%%%%%%%%%%%%%%%%%%%
