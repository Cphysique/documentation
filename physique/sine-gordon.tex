%
%%%%%%%%%%%%%%%%%%%%%
\section{L'équation de Sine-Gordon}
%%%%%%%%%%%%%%%%%%%%%
%
Cette section traite de l'équation de sine-gordon et de ses solutions
\subsection{L'équation de Sine-Gordon}
C'est une équation différentielle du second ordre, non linéaire, à deux variables \cite{sine-gordon}.
\[
\frac{\partial^2\theta}{\partial t^2} - c^2 \frac{\partial^2\theta}{\partial x^2} + \omega _0 ^2 \sin \theta = 0
\]
%
%
%
\subsection{Les solitons, solutions de l'équation de Sine-Gordon}
Une solution de l'équation de Sine-Gordon, appelée soliton, est 
\[
\theta(x,t)=4\arctan \exp ( \omega t - \text{k} x )
\]
%\[ \mbox{Let } x = \mbox{ number of cats} \]%\text{$$}
%\[ \textrm{Let } x = \textrm{ number of cats} \]
Elle correspond à une variation de 2$\pi$ de la valeur de $\theta$ sur une distance de l'ordre de k$^{-1}$. Le soliton se déplace à la vitesse v.
%c=$\omega / k $
\subsection{Phénomènes physiques associés aux solitons}
La {\bf jonction josephson}. Constitué par une jonction isolante entre deux supraconducteur.

Les {\bf motifs du pelage des animaux} \cite{pelage-animaux}

Les {\bf frontières}. New-York possède un quartier chinois et un quartier italien. Le grignotage de Little Italy par Chinatown montre le déplacement de la brutale variation des densités de population chinoise et italienne.

%%%%%%%%%%%%%%%%%%%%%%%%%%%%%%%%%%%%%%%%%%%%%%%%%%%%%%%%%%%%%%%%%%%%%%%%%%%%%%%%%%%%%
