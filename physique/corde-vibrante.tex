%
%%%%%%%%%%%%%%%%%%%%%
\section{La corde vibrante}
%%%%%%%%%%%%%%%%%%%%%
%
Cette section traite de l'équation des cordes vibrantes, de ses solutions et de la transformée de fourier.
\subsection{La corde vibrante}
\label{boutin}
C'est une équation différentielle du second ordre, linéaire, à deux variables \cite{corde-vibrante}.
\[
\frac{\partial^2 u}{\partial t^2} - c^2 \frac{\partial^2 u}{\partial x^2} = 0
\]
%
%
%
\subsection{La transformée de fourrier}
\label{fourier}
%
La transformée de fourier correspond à la décomposition d'une fonction sur une base de fonctions harmoniques. \cite{fourier}
\[
\hat{u}(k) = \int_{-\infty}^{+\infty} u(x) \mathrm e^{-2 i \pi k x} \, \mathrm dx
\]
%
%
%
%%%%%%%%%%%%%%%%%%%%%%%%%%%%%%%%%%%%%%%%%%%%
%%%%%%%%%%%%%%%%%%%%%%%%%%%%%%%%%%%%%%%%%%%%%%%%%%%%%%%%%%%%%%%%%%%%%%%%%%%%%%%%%%%%%
