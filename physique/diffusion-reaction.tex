%
%%%%%%%%%%%%%%%%%%%%%
\section{Les équations de Turing-Murray}
%%%%%%%%%%%%%%%%%%%%%\nabla
%
Cette section traite du système d'équations de Turing-Murray

Les {\bf motifs du pelage des animaux} \cite{pelage-animaux} \label{vidiani}
 \cite{motifs-pelage} \label{bejjani}

C'est un système de deux équations différentielle du second ordre, non linéaire, à trois variables  \cite{pelage-animaux} \label{vidiani}.
\[
\left\{
   \begin{array}{r c l}
      \frac{\partial{\bf u}}{\partial t}  & = &  \gamma f({\bf u},{\bf v}) + D_1\nabla^2{\bf u} \\
      \frac{\partial{\bf v}}{\partial t}  & = &  \gamma g({\bf u},{\bf v}) + D_2\nabla^2{\bf v}
   \end{array}
\right.
\]

%
\subsection{Phénomènes physiques décrit par des équations non linéaires}


Les {\bf frontières}. New-York possède un quartier chinois et un quartier italien. Le grignotage de Little Italy par Chinatown montre le déplacement de la brutale variation des densités de population chinoise et italienne.

%%%%%%%%%%%%%%%%%%%%%%%%%%%%%%%%%%%%%%%%%%%%%%%%%%%%%%%%%%%%%%%%%%%%%%%%%%%%%%%%%%%%%
