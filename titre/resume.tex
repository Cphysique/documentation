\begin{center}
\Large
Résumé
\normalsize
\end{center}
\vspace{3cm}
\begin{itemize}[leftmargin=1cm, label=\ding{32}, itemsep=21pt]
\item {\bf Objet : }Ce document (en cours de construction), accompagne les programmes SiCP, SiCF et SiGP (eux mêmes en cours de développement).
\item {\bf Contenu : }Il contient un manuel d'installation et d'utilisation ainsi que quelques développements théoriques liés à ces programmes de simulations numériques.
\item {\bf Public concerné : }Ce document s'adresse aux enseignants et aux étudiants du supérieur des sections sciences physiques et informatique.
\end{itemize}

\vspace{3cm}

SiCP, SiCF et SiGP sont des simulateurs numériques d'équations physiques offrant une représentation graphique et une interaction dynamique avec les paramètres physiques. Destinés à un usage pédagogique, ils permettent de visualiser le comportement des systèmes physiques simulés. Cette documentation accompagne ces programmes.

\begin{itemize}[leftmargin=1cm, label=\ding{32}, itemsep=11pt]
\item Les deux premiers chapitres présentent les simulateurs, fournissent une procédure d'installation et précisent les commandes permettant l'interaction avec les programmes.
\item Les deux chapitres suivants fournissent un certain nombre de développements théoriques liés au phénomènes physiques et à la numérisation des équations.
\item Enfin, le dernier chapitre rassemble les informations liées à la structure des programmes.
\end{itemize}
